\documentclass{beamer}
\usepackage{graphics}
\usepackage{epsfig}
\usepackage{multicol}
\usepackage{pifont}
\setbeamertemplate{navigation symbols}{}
\newcommand{\RR}{\ensuremath{\mathbb{R}}}
\newcommand{\NN}{\ensuremath{\mathbb{N}}}
\newcommand{\QQ}{\ensuremath{\mathbb{Q}}}
\newcommand{\CC}{\ensuremath{\mathbb{C}}}
\newcommand{\ZZ}{\ensuremath{\mathbb{Z}}}
\newcommand{\TT}{\ensuremath{\mathbb{T}}}
\newcommand{\HH}{\ensuremath{\mathbb{H}}}
\DeclareMathOperator{\Min}{Min}
\DeclareMathOperator{\mint}{min}
\DeclareMathOperator{\vertt}{vert}
\DeclareMathOperator{\conv}{conv}
\DeclareMathOperator{\rank}{rank}

\def\QuotS#1#2{\leavevmode\kern-.0em\raise.2ex\hbox{$#1$}\kern-.1em/\kern-.1em\lower.25ex\hbox{$#2$}}

\begin{document}
\title{Wave models}
\author{
\begin{center}
\textcolor{red}{\large Mathieu Dutour Sikiri\'c}\\[2mm]
\textcolor{red}{Rudjer Bo\u skovi\'c Institute, Croatia}\\[2mm]
\end{center}
}

\date{\today} 
\frame{\titlepage} 

\frame{
  \frametitle{Wave models available}
\begin{itemize}
\item {\bf SWAN}: Available as GPL, coupled with ROMS and other models.
\item {\bf WaveWatch III}: From NOAA and IFREMER. It has a large arrays of possibilities, but is more complex to use
\item {\bf WAM}: Developped at ECMWF. 
\item {\bf WWM}: The Wind Wave Model developped by Aron Roland and Mathieu Dutour Sikiri\'c.
\end{itemize}
All models support unstructured meshes. Only WWM has no structured grids.
}


\frame{
  \frametitle{The WWM model}
The Wind Wave Model is a third generation wave model authored by Aron Roland and which shares many
common features with WaveWatch III.
\begin{itemize}
\item The Wind Wave Model (WWM) is a unstructured grid spectral wave model.
\item It incorporates most existing source term formulation for wind input and dissipation (Cycle III,
Cycle IV, Ardhuin, Makin, ...)
\item It has been coupled to SELFE, SHYFEM, TIMOR and ROMS.
\item It uses Residual Distribution schemes for the horizontal advection.
\item It integrates the WAE by using the Operator Splitting Method in explicit or implicit mode.
\item It has NETCDF output/input/hotfile.
\item Parallelization is done by ParMETIS.
\end{itemize}
}

\frame{
  \frametitle{Model coupling, physics}

\begin{itemize}
\item The coupling can be done {\bf Wave} - {\bf Atmosphere}:
\begin{itemize}
\item The atmosphere sends wind speed, air density, stability terms to the wave model
\item The wave model sends Charnock coefficient to the Atmosphere for the computation of the surface stress.
\end{itemize}
\item The coupling can be done {\bf Wave} - {\bf Ocean}:
\begin{itemize}
\item Currents and surface are send to the wave model
\item Wave dissipation can be used for parameterization of turbulence.
\item radiation stress enter into the primitive equations (Ardhuin formulation, etc.)
\item wave parameters can be used for sediment formulations.
\end{itemize}
\item The coupling can be done {\bf Atmosphere} - {\bf Ocean}:
\begin{itemize}
\item The ocean model receives wind, air temperature, humidity from the atmosphere.
\item The atmospheric model receives the sea surface temperature.
\end{itemize}


\end{itemize}
}



\frame{
  \frametitle{Model coupling library, computer science}
\begin{itemize}
\item The exchange between coupled models (via {\tt COMM\_SPLIT}) requires the sending of data between them.
\item A priori the grids are different, the model nature may be different (Structure/Unstructured grids)
and so interpolation is needed between the models.
\item There are several existing libraries {\tt MCT}, {\tt OASIS}, {\tt PALM}, etc but when considering
them, they appear all relatively complicated.
\item We designed our own simple library for coupling {\tt PGMCL} (Parallel Geophysical Model Coupling Library)
\item After declarations, the commands become as simple as
\begin{center}
{\tt CALL MPI\_INTERP\_SEND(TheArr\_WAVtoOCN, Hwave)}\\
{\tt CALL MPI\_INTERP\_RECV(TheArr\_WAVtoOCN, Hwave)}
\end{center}
\item We coupled {\bf COSMO}, {\bf WAM}, {\bf ROMS} and {\bf WWM} together 
\end{itemize}
}

\frame{
  \frametitle{Data comparison}
\begin{itemize}
\item Right now no wave model has data assimilation implemented in it.
\item Classical assimilation methods such as Kalman filter and 4DVAR depends on linearization assumptions,
which are not valid there.
\item Instead, primary tool is to compare model results with measurement or satellite estimate.
\item {\bf Altimeter estimates} of significant wave height can be used.
\item {\bf Buoys measurements} can be used. Variables:
\begin{itemize}
\item Significant wave height.
\item wave periods
\item other turbulence parameters.
\end{itemize}

\end{itemize}
}








\end{document}
