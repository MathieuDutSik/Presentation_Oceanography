\documentclass[%
pdf,
%nocolorBG,
colorBG,
slideColor,
%slideBW,
%draft,
%frames
%azure
%contemporain
%nuancegris
%troispoints
%lignesbleues
%darkblue
%alienglow
%autumn
]{prosper}
\usepackage{pifont, amsmath, multicol}
%\usepackage{floatflt, wrapfig,subfigure}
\usepackage{color}
\usepackage{epsfig}
\usepackage{pifont}
%\usepackage[francais]{babel}
\usepackage[T1]{fontenc}
\usepackage[latin1]{inputenc}
%\catcode`\�=\active
%\catcode`\�=\active
%\def�{\og\ignorespaces}%
%\def�{{\fg}}%
\newcommand{\spacer}{\rule[-3mm]{0mm}{8mm}}
\newtheorem{theorem}{Theorem}
\newtheorem{definition}[theorem]{Definition}
\newtheorem{conjecture}[theorem]{Conjecture}
\newtheorem{corollary}[theorem]{Corollary}
\newtheorem{problem}[theorem]{Problem}
\newtheorem{lemma}[theorem]{Lemma}
\newtheorem{claim}[theorem]{Claim}
\newtheorem{remark}[theorem]{Remark}
\newtheorem{proposition}[theorem]{Proposition}


\newcommand{\RR}{\ensuremath{\mathbb{R}}}
\newcommand{\NN}{\ensuremath{\mathbb{N}}}
\newcommand{\QQ}{\ensuremath{\mathbb{Q}}}
\newcommand{\CC}{\ensuremath{\mathbb{C}}}
\newcommand{\ZZ}{\ensuremath{\mathbb{Z}}}
\newcommand{\TT}{\ensuremath{\mathbb{T}}}



\title{{\Huge \textcolor{blue}{Kalman filtering}\\
\textcolor{blue}{in}\\[5mm]
\textcolor{blue}{oceanography}}}
\author{
\textcolor{red}{\Large Mathieu Dutour Sikiri\'c}\\[2mm]
\textcolor{red}{\Large (based on Hoteit Ibrahim thesis)}\\[2mm]
%\textcolor{red}{\large ENS/CNRS, Paris and Hebrew University, Jerusalem}\\[2mm]
}
\slideCaption{}

\date{}



\begin{document}
\maketitle


\begin{slide}{Problem}

\begin{itemize}
\item An evolution equation with some uncertainties:
\begin{itemize}
\item incomplete modelization of the system
\item incorrectness of the numerical model.
\end{itemize}
\item Some measurement with some uncertainty.
\item[\textcolor{blue}{\ding{224}}] combine this to get good estimate on the ``true'' state of the system.
\end{itemize}

\end{slide}






\begin{slide}{History}
{\scriptsize
\begin{itemize}
\item[\textcolor{blue}{\ding{224}}] (1949) \textcolor{red}{Kolmogorov - Wiener}, spectral decomposition (under assumption of linear autonomous system)
\item[\textcolor{blue}{\ding{224}}] (1960) \textcolor{red}{Kalman - Bucy}, sequential filtering for linear system.
\item[\textcolor{blue}{\ding{224}}] (1994) \textcolor{red}{Evensen}, Ensemble Kalman filtering
\end{itemize}
}
\end{slide}










%%%%%%%%%%%%%%%%%%%% Slide presentation
\begin{slide}{}
\begin{center}
{\Huge 
\begin{tabular*}{6cm}{c}
\\[-0.5cm]
\textcolor{blue}{I. }\textcolor{red}{Original}\\
\textcolor{red}{Kalman}\\
\textcolor{red}{filtering}
\end{tabular*}
}
\end{center}
\end{slide}


\begin{slide}{Gaussian variables}
\begin{itemize}
\item A random variable $X$ is Gaussian if its probability density is proportional to
\begin{equation*}
exp(-\frac{1}{2\sigma}(x-m)^2)
\end{equation*}
\item Gaussian random variable are characterized by
\begin{equation*}
E(X)=m\mbox{~~~and~~~}E(X^2)=\sigma+m^2
\end{equation*}
\item The Gaussian Random variable appearing in Oceanography are of course \textcolor{red}{multi-dimensional}. They are characterized by a \textcolor{red}{vector} and a \textcolor{red}{covariance matrix}.
\end{itemize}

\end{slide}





\begin{slide}{Stochastic Differential equations}
\begin{itemize}
\item Stochastic differential equation:
\begin{equation*}
dX_t=F(t)X_t dt+C(t)dU_t
\end{equation*}
with $dU_t$ a ``white noise'', i.e. is a Brownian motion, Gaussian process.
It is the \textcolor{red}{model equation}.
\item Stochastic differential equation:
\begin{equation*}
dZ_t=G(t)X_t dt+ D(t)dV_t
\end{equation*}
with $dV_t$ another independent ``white noise''.
It is the \textcolor{red}{measurement equation}.
\end{itemize}
At every given $t$, $X_t$, $Z_t$ will be Gaussian variables.
\end{slide}





\begin{slide}{Kalman solution}
\begin{itemize}
\item The stochastic equations are considered to be \textcolor{red}{Ito equations} (the alternatice is Stratonovich equations).
\item Kalman-Bucy theory allows to compute
\begin{itemize}
\item the expectancy,
\item covariance matrix 
\end{itemize}
\item But they are solution of a nonlinear Riccati ordinary differential equation.
\end{itemize}
\end{slide}



\begin{slide}{Discrete version}
\begin{itemize}
\item Evolution equation:
\begin{equation*}
X^{t}(t_k)=A_k X^t(t_{k-1})+\eta(t_k)
\end{equation*}
with $\eta(t_k)$ being Gaussian of covariance matrix $Q(t_k)$
\item Measurement equation
\begin{equation*}
Y_k^{0}=H_k X^t(t_k)+\epsilon_k
\end{equation*}
with $\epsilon_k$ being Gaussian of covariance matrix $R(t_k)$
\end{itemize}
All $X^t$ terms appearing are Gaussian.
\end{slide}



\begin{slide}{The Equations}


\begin{itemize}
\item {\bf Initialisation step}
\begin{equation*}
\begin{array}{rcl}
X^a(t_0)&=&m_0\\
P^a(t_0)&=&P_0
\end{array}
\end{equation*}
\item {\bf Prevision step}
\begin{equation*}
\begin{array}{rcl}
X^f(t_k)&=&A_k X^a(t_{k-1})\\
P^f(t_k)&=&A_k P^a(t_{k-1})A_k^T+Q(t_k)
\end{array}
\end{equation*}
\item {\bf Correcting step}
\begin{equation*}
\begin{array}{rcl}
X^a(t_k)&=&X^f(t_k)+K_k\{Y_k^0-H_k X^f(t_k)\}\\
P^a(t_k)&=&(I-K_kH_k)P^f(t_k)
\end{array}
\end{equation*}
with $K_k=P^f(t_k) H_k^T\{H_kP^f(t_k)H_k^T+R_k\}^{-1}$.
\end{itemize}
\end{slide}


\begin{slide}{Derivation of the equations}
Denote $X^t$ and $Y^t$ the true state and observation of the system at instant $t_k$
\begin{itemize}
\item We want to minimize the expectancy 
\begin{equation*}
\begin{array}{rcl}
&=&E[(X^a-X^t)(X^a-X^t)^{T}]\\
&=&E[(X^f-X^t+K(Y^0-HX^f))(\dots)^T]\\
&=&E[((I-KH)(X^f-X^t)+K(Y^0-Y^t))(\dots)^T]\\
&=&(I-KH)E[(X^f-X^t)(X^f-X^t)^T](I-KH)^T\\
&&+KE[(Y^0-Y^t)(Y^0-Y^t)^T]K^T\\
&=&(I-KH)P^f(I-KH)^T+KRK^T
\end{array}
\end{equation*}
\item Minimization is obtained by $K=P^f(t_k) H_k^T\{H_kP^f(t_k)H_k^T+R_k\}^{-1}$.

\end{itemize}
\end{slide}


\begin{slide}{}
\begin{center}
{\Huge 
\begin{tabular*}{8cm}{c}
\\[-0.5cm]
\textcolor{blue}{II. }\textcolor{red}{Extensions}\\
\textcolor{red}{and}\\
\textcolor{red}{restrictions}
\end{tabular*}
}
\end{center}
\end{slide}


\begin{slide}{Extended Kalman filtering}
\begin{itemize}
\item Equations for \textcolor{red}{nonlinear} systems:
\begin{equation*}
\begin{array}{rcl}
X^{t}(t_k)&=&F_k(X^t(t_{k_1}))+\eta(t_k)\\
Y_k^{0}&=&H_k X^t(t_k)+\epsilon_k
\end{array}
\end{equation*}
\item Kalman filtering applies only to \textcolor{red}{linear}\mbox{~} systems. 
\item \textcolor{red}{nonlinear}$\Rightarrow$ \textcolor{red}{Gaussianity} no longer preserved.
\item Extended Kalman filtering is a linearization of the equations.
\item Consequences and results:
\begin{itemize}
\item We lose optimality of Kalman filtering
\item It works relatively well for weakly nonlinear systems.
\item It does not work well for strongly nonlinear systems.
\end{itemize}


\end{itemize}
\end{slide}


\begin{slide}{Issues for oceanography}
\begin{itemize}
\item In order to apply Extended Kalman filtering, one needs:
\begin{itemize}
\item the variance matrices of measurement.
\item have an estimation of ``what we miss''
\item have an estimation of the computer errors.
\end{itemize}
\item It is not possible to invert matrices of size $1.10^6\times 1.10^6$.
\item[\ding{224}] Kalman filtering or Extended Kalman filtering \textcolor{red}{cannot}\\
be applied in Oceanography.
\end{itemize}
\end{slide}



\begin{slide}{The SEEK filter}
\textcolor{blue}{SEEK}: Singular Evolutiv Extended Kalman Filter
\begin{itemize}
\item Idea is that the phase state has a variety called \textcolor{red}{attractor}.
\begin{itemize}
\item \textcolor{red}{normal} directions to the attractor are \textcolor{red}{dissipative}: errors are corrected themselves.
\item \textcolor{red}{tangent} directions to the attractor are \textcolor{red}{hyperbolic}: errors are amplified.
\end{itemize}
\item \textcolor{blue}{SEEK}: we consider the error only in a chosen tangent space of dimension $r$.
\item The computational cost is $(r+1)$ times the cost of the model.

\end{itemize}
\end{slide}





\begin{slide}{The \textcolor{blue}{SEIK} filter}
\textcolor{blue}{SEIK}: Singular Extended Interpolated Kalman Filter
\begin{itemize}
\item Take a cloud of $r+1$-points around the initial state and make them evolve.
\item Compute the initial state and its covariance matrix by 
\begin{equation*}
P^a(t_{k})=\frac{1}{r+1}\sum_{i=1}^{r+1}[X_i^a(t_k)-X^a(t_k)][\dots]^T
\end{equation*}
\item cost is identical to the one of \textcolor{blue}{SEEK}, performance are superior; one possible reason: linearization implies a bigger error than doing averaging.
\end{itemize}
\end{slide}


\begin{slide}{The subspace in \textcolor{blue}{SEIK} and \textcolor{blue}{SEEK}}
\begin{itemize}
\item In order to make those filters run, one needs to select a basis.
\item Hoteit uses
\begin{itemize}
\item Empirical Orthogonal Functions.
\item Localized EOF around areas of interest.
\end{itemize}
\item Empirical orthogonal functions consist of having $N$ states and selecting an orthogonal basis of $r$ states ($r< N$) that encapsulates the main trends.
\item localized EOF consist of selecting some areas, and localizing the functions on those areas, then doing classical EOF.
\item ``mixed'' basis consist of localized EOF + some global EOF. In \textcolor{blue}{SESEEK} he makes evolve only the global EOF.

\end{itemize}
\end{slide}



\begin{slide}{Computationnal issues}
\begin{itemize}
\item \textcolor{blue}{SEEK} and \textcolor{blue}{SEIK} are possible to use in Oceanography
\item their cost remain high.
\item Further ideas are needed...
\end{itemize}
\end{slide}



\begin{slide}{\textcolor{blue}{SFEK} and \textcolor{blue}{SAEK} filters}
\begin{itemize}
\item If we fix the basis of the space in which computations are done at the beginning, we obtain the \textcolor{blue}{SFEK} filter.
\item In \textcolor{red}{linear autonomous} case, the matrix $P^a$ converges to a fixed matrix.\\
\textcolor{blue}{\ding{224}} The idea is to fix the subspace considered, \textcolor{blue}{SAEK}.
\item Hoteit indicates some poor behaviour, due to the fact that Ocean is nonlinear.
\end{itemize}
\end{slide}





\begin{slide}{SIEIK}
Singular Intermittemt Extended Interpolated Kalman Filter
\begin{itemize}
\item \textcolor{blue}{SFEK} and \textcolor{blue}{SAEK} are problematic, since the base is evolving.
\item[\textcolor{blue}{\ding{224}}] The idea is to make the base evolves by intermittence:
\begin{itemize}
\item if the model stays calm, we keep the same basis,
\item if some significant modification happen, we update the basis.
\end{itemize}
\item It makes a faster filter and it works almost as good as \textcolor{blue}{SEEK}.
\end{itemize}
\end{slide}











\begin{slide}{}
\begin{center}
{\Huge 
\begin{tabular*}{8cm}{c}
\\[-0.5cm]
\textcolor{blue}{III. }\textcolor{red}{Ensemble}\\
\textcolor{red}{Kalman}\\
\textcolor{red}{filtering}
\end{tabular*}
}
\end{center}
\end{slide}






\begin{slide}{The idea}
\begin{itemize}
\item The idea is to replace the formalism of average and covariance matrix by a population of states ($O(100)$ states in Evensen experiments)

\item[\textcolor{blue}{\ding{224}}] the system can handle nonlinear situations better.

\item Its cost is similar to the one of \textcolor{blue}{SEEK}, \textcolor{blue}{SEIK}.

\item The overall idea is to do Markov Chain Monte Carlo simulation to make evolve the net of points.

\end{itemize}
\end{slide}







\begin{slide}{The error terms }
\begin{itemize}
\item The uncertainty on measurement is handled by the ensemble.
\item What about the uncertainty of the model? It is still assumed to
be a linear white noise of the form
\begin{equation*}
\psi_k=f(\psi_{k-1})+q_k
\end{equation*}
\item Despite this the \textcolor{blue}{EnKF} performs very well and manages to work even on Lorenz model.

\end{itemize}
\end{slide}









\begin{slide}{}
\begin{center}
{\Huge 
\begin{tabular*}{4cm}{c}
\\[-0.5cm]
\textcolor{red}{Thank}\\
\textcolor{red}{You}
\end{tabular*}
}
\end{center}
\end{slide}


\end{document}
