\documentclass{beamer}
\usepackage{graphics}
\usepackage{epsfig}
\usepackage{multicol}
\usepackage{pifont}
\setbeamertemplate{navigation symbols}{}
\newcommand{\RR}{\ensuremath{\mathbb{R}}}
\newcommand{\NN}{\ensuremath{\mathbb{N}}}
\newcommand{\QQ}{\ensuremath{\mathbb{Q}}}
\newcommand{\CC}{\ensuremath{\mathbb{C}}}
\newcommand{\ZZ}{\ensuremath{\mathbb{Z}}}
\newcommand{\TT}{\ensuremath{\mathbb{T}}}
\newcommand{\HH}{\ensuremath{\mathbb{H}}}
\usepackage[version=4]{mhchem}
\DeclareMathOperator{\Min}{Min}
\DeclareMathOperator{\mint}{min}
\DeclareMathOperator{\vertt}{vert}
\DeclareMathOperator{\conv}{conv}
\DeclareMathOperator{\rank}{rank}

\def\QuotS#1#2{\leavevmode\kern-.0em\raise.2ex\hbox{$#1$}\kern-.1em/\kern-.1em\lower.25ex\hbox{$#2$}}

\begin{document}


\title{Modelling of Rogoznica lake: Physics and Biology}


\author{
\textcolor{red}{\large Mathieu Dutour Sikiri\'c}\\[2mm]
\textcolor{black}{Institute Rudjer Bo\u skovi\'c}\\
}

\date{\today}
\frame{\titlepage}


\frame{
  \frametitle{Statement of the problem}

\begin{itemize}
\item Rogoznica lake is a lake that is either in one of two states:
\begin{itemize}
\item In a straitified state with an Oxic layer in the upper part, a chemocline lower and then a lower anoxic layer.
\item An unstratified state where the whole column is anoxic.
\end{itemize}
\item The following aspects are important to the modelization
\begin{itemize}
\item The concentrations in temperature/salinity are varying in time and govern the strength of the stratification.
\item The sulfur chemistry in the lake is complex with $SO_4^{2-}$, $S^{2-}$ and $S^0$ being the most significant part.
\item The primary production of matter is by the phytoplankton and the organic matter is decomposed by heterotrophic bacteria.
\end{itemize}


\end{itemize}
}


\frame{
\begin{center}
\begin{tabular*}{6cm}{c}
\\[-0.5cm]
{\Huge \textcolor{blue}{I. }\textcolor{red}{Physical}}\\
{\Huge \textcolor{red}{description}}
\end{tabular*}
\end{center}
}



\frame{
  \frametitle{Fundamental simplying assumption}

\begin{itemize}
\item We choose to assume that the physics determine the biology and that the physics is determined very little by biology
\item The immediate consequence is that we can separate the modelling into two stages:
\begin{itemize}
\item Modeling the physical parameters of the lake.
\item Modeling of the chemistry and biology separately.
\end{itemize}
This is a very common approach. For example one commonly run atmospheric model and then feed the obtained wind, temperature and other fields to the oceanographic model.
\item Where the assumption could be wrong is that biological activity could change where light is absorbed which would then affect temperature.
\end{itemize}
}






\frame{
  \frametitle{Air sea exchanges}
\begin{itemize}
\item The following gases are exchanged at the surface:
\begin{itemize}
\item Oxygen $O_2$ which is consumed by respiration and created by phytoplankton
\item Carbon dioxide $CO_2$ which is consumed by phytoplankton and created by respiration.
\item hydrogen sulfide $H_2S$ which is created by the heterotrophic bacteria and consumed by the photopic bacteria and dissolved by oxygen
\end{itemize}
\item Henry's law specifies the concentration of a gas at the surface in terms of the partial pressure of the gas at the surface. In practice:
\begin{itemize}
\item Oxygen and carbon dioxide concentration are fixed by the concentrations in the atmospheric pressure.
\item Hydrogen sulfide that reach the surface go to the atmosphere.
\end{itemize}
More sophisticated parametrization of this equilibrium exist.
\item Some parameterization of the dynamic evolution in time uses the wind speed.
\end{itemize}
}



\frame{
  \frametitle{Limnic eruptions}

\begin{itemize}
\item Limnic eruptions occur when the concentration in gas reaches too high a concentration to remain dissolved and bubbles form
\item When the bubbles form, they go upward where the pressure is lower and so more gas gets released which causes an acceleration.
\item When this occurs in some lakes in Africa it can reach heights of $80$ meters which then releases the gas in the atmosphere.
\item There is no such risk in Rogoznica lake.
\end{itemize}
}



\frame{
  \frametitle{Physical modelization and Diffusion in the lake}

\begin{itemize}
\item A predominant problem that we have to resolve is the diffusion of the chemicals in the lake.
\item Currents and water level variations contribute to the turbulent kinetic energy of the lake but their magnitude is relatively small.
\item The diffusion coefficient is parameterized via the strength of the parameterization:
\begin{equation*}
K_z = K_{z, back} (1 + N_2^2)^{-2} \mbox{ with } N_2 = \frac{d\rho}{dz} \frac{g}{\rho}
\end{equation*}
\item This means that when the water column is uniform, the diffusion coefficient is very high but that when the density stratification
  is very strong with respect to the density then the diffusion coefficient is low.
\item This implies that the oxic and anoxic part of the lake became areas with relatively uniform concentrations but that the stratification
  separates the two zones.
\end{itemize}
}





\frame{
\begin{center}
\begin{tabular*}{6cm}{c}
\\[-0.5cm]
{\Huge \textcolor{blue}{II. }\textcolor{red}{Chemical and}}\\
{\Huge \textcolor{red}{biological aspects}}
\end{tabular*}
\end{center}
}



\frame{
  \frametitle{$pH$ parameterization}

\begin{itemize}
\item The $pH$ is affected by a lot of substrate in the lake.
\item How important is the $pH$?
\begin{itemize}
\item The chemical reactions mediated by the bacteria are changing the $pH$
\item The $pH$ affects the reaction and their speed.
\end{itemize}
\item But the $pH$ is influenced by many processes in the lake which makes its modelization difficult.
\item On the other hand, measurement of the $pH$ in the lake shows that the variability is not very large.
\item Thus we choose to set the $pH$ to a constant value of $7$.
\end{itemize}
}




\frame{
  \frametitle{The sulfur cycle in the lake}
\begin{itemize}
\item The sulfate $SO_4^{2-}$ is processed by heterotrophic bacteria by
\begin{center}
\ce{2 H^{+} + SO_4^{2-} + \{ CH_2O \}  -> H2S + 2CO2 + 2H2O}
\end{center}
\item The hydrogen sulfide $H_2S$ is a poison to the bacteria including the heterotropic bacteria.
\item What is the fate of the $H_2S$ in the lake:
\begin{itemize}
\item Some can be processed by the phototropic bacteria according to
\begin{center}
\ce{2 H2S + CO2 -> \{ CH_2O \}  + 2S + H2O}
\end{center}
\item The elementary sulfur can react to form polysulfides
\item Some can be disproportionated by the reaction
\begin{center}
\ce{4 S + 4 H2O -> 3 H2S + SO_4^{2-} + 2H+}
\end{center}
\item But eventually some of the hydrogen sulfide had to seep into the upper layer and react with the oxygen to transform into sulfate.



\end{itemize}
\end{itemize}
}




\frame{
  \frametitle{Phytoplankton}

\begin{itemize}
\item Living organisms are very complex and in order to model them we have to make big modelling choice:
\begin{itemize}
\item Only one single specie is considered.
\item Organic matter itself is just encoded by the single symbol $\{ CH_2O \}$.
\item Dependency upon nutrients concentration $[C]$ is simply expressed as $[C] / (K + [C])$
\item There is also a dependency on the light.
\item Some terms account for the overpopulation.
\item The death rate is taken constant.
\end{itemize}
\item The modelization is thus expressed as
\begin{equation*}
\frac{dB}{dt} = \gamma_{max} f_{light} f_{pop} f_{nutrient} B - c_{death} B
\end{equation*}



\end{itemize}
}







\frame{
  \frametitle{Heterotrophic / Phototropic / Disproportionating bacteria}

\begin{itemize}
\item {\bf Heterotrophic bacteria} live off the organic matter (DOC/POC) that falls into the bottom of the lake when dying.
\item The modelization depends on:
\begin{itemize}
\item The concentration of POC/DOC.
\item The concentration in $H_2S$ which is toxic to the bacteria
\end{itemize}
\item The modelization is thus expressed as
\begin{equation*}
\frac{dH}{dt} = \gamma_{max} f_{H_2S} H - c_{death} H
\end{equation*}
\item {\bf Phototropic bacteria} uses photosynthesis to make elementary sulfur out of $H_2S$. Their modelization depends on
\begin{itemize}
\item The concentration of $H_2S$
\item The light available.
\end{itemize}
\item {\bf Disproportionating bacteria} uses elementary sulfur to create $SO_4^{2-}$ and $H_2S$. The reaction is very slow in nature and bacteria eked out a living out of it. Their parametrizing is very difficult.



\end{itemize}
}




\frame{
  \frametitle{Summary}
\begin{itemize}
\item Hydrogen sulfide $H_2S$ seeps into the oxic part and is converted to $SO_4^{2-}$. The absence of hydrogen sulfide is not the absence of hydrogen sulfide arriving there.
\item The concentration in nutrients is higher in the bottom of the lake because that is where the POC/DOC falls.
\item The concentration in phytoplankton is high nearer the chemocline because this is where the nutrients are.
\end{itemize}
}




\end{document}
